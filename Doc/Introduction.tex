\documentclass[12pt,titlepage]{article}
\usepackage{amssymb}
\usepackage {latexsym}
\usepackage{amsmath}
\usepackage{hyperref}
\usepackage{verbatim} 
\usepackage{graphicx}
\usepackage{epstopdf}

\title{\LARGE{\textbf{ChaRM python package}} \\[10pt] 
\normalsize{Version 0.1.0}}
\author{ \textbf{\href{mailto:mmaysami@eos.ubc.ca}{Mohammad
      Maysami}}\\[15pt] 
\normalsize{\href{http://slim.eos.ubc.ca}{Seismic Laboratory for
    Imaging and Modeling~(SLIM)}}\\ 
\normalsize{\href{http://www.eos.ubc.ca}{Department of Earth and Ocean
    Sciences(EOSC)}}\\ 
\normalsize{\href{http://www.ubc.ca}{University of British Columbia (UBC)}}\\ }
\date{December 2007}
%=============================
%  	    Main Body
%=============================
\begin{document}
\maketitle
%
%
\section*{Description}
\textit{Charm} is a python package to implement recent studies on
ChaRM project (Characterization of reflectors and modeling). It is
implementing the new trace-based algorithm we proposed in our recent
abstract
(\href{http://slim.eos.ubc.ca/?module=articles&func=display&ptid=10&aid=119}{Maysami  and Herrmann, 2007} 
\footnote{M. Maysami and F.J. Herrmann, Seismic
  reflector characterization by a multiscale detection-estimation
  method, accepted for EAGE 69th Conference \& Exhibition, London.})
%
%
\section*{License}
You may use this code only under the conditions and terms of the
license contained in the file \textit{LICENSE} provided with this
source code. If you do not agree to these terms you may not use this
software.
%
%	
\section*{Prerequisites}
\begin{itemize}
\item Python-2.4 or newer
  \textit{(\href{http://www.python.org/}{\small{http://www.python.org}})}

\item NumPy-1.0 or newer
  \textit{(\href{http://numpy.scipy.org/}{\small{http://numpy.scipy.org}})}

\item SciPy-0.5.1 or newer
  \textit{(\href{http://www.scipy.org/}{\small{http://www.scipy.org}})}

\item Matplotlib-0.90.1 or newer
  \textit{(\href{http://matplotlib.sourceforge.net/}{\small{http://matplotlib.sourceforge.net}})}

\item MADAGASCAR-2817 \textit{(SVN developer tree at
    \href{http://rsf.sourceforge.net/}{\small{http://rsf.sourceforge.net}})}

\quad Note: Compile with API=python

% \item SCons-0.96.95 or newer
%   \textit{(\href{http://www.scons.org/}{\small{http://www.scons.org}})}

\quad Note: It is optional and only required to run SConstruct script.
\end{itemize}
%
%
\section*{Installation}
There are two easy installation method for this package. Considering
permissions limits for non-administrator users, the first one is
advised since it use this package in the place you have it copied.
\begin{itemize}

\item Just put the package folder (Charm) into your
  \texttt{\$PYTHONPATH}. For example, if you are using
  \small{\textit{BASH}} and \texttt{\$Charm} is the path to your Charm
  folder(excluding the folder itself), then type:
	\begin{verse}
          \texttt{export PYTHONPATH=\$PYTHONPATH:\$Charm}
	\end{verse}
        \quad For \textit{CShell} type:
	\begin{verse}
          \texttt{setenv PYTHONPATH \${PYTHONPATH}:\$Charm}
	\end{verse}

      \item Alternatively, you may use setup.py to install the package
        in your default python path location, enter this command:
	\begin{verse}
		\texttt{python setup.py install}
	\end{verse}

        To install it into a specific folder type this, where
        /path/to/mypython is a folder which is in your
        \texttt{\$PYTHONPATH} environment.
% 	\begin{verse} \texttt{python setup.py install --install-lib=/path/to/mypython}
% 	\end{verse}

      \item Set environment variables(paths) if they are different
        than default values. These variables are
        \emph{CHARM\_Data,CHARM\_Results,CHARM\_pydata,CHARM\_Demos}. Otherwise
        it will be set to defaults which are folders with the same
        name (Data, Results, and etc.) in \texttt{\$Charm} as
        mentioned above.
  
For example, if you are using \small{\textit{BASH}} and \texttt{\$DataPATH} is the path where you have input rsf data,then type:
%
\begin{verse}
  \texttt{export CHARM\_Data=\$DataPATH}
\end{verse}
\quad For \textit{CShell} type:
\begin{verse}
  \texttt{setenv CHARM\_Data \$DataPATH}
\end{verse}
%
\item One of the global variables,  \texttt{\_df\_input}, in \texttt{\_\_init\_\_.py} points to default input rsf file for some of the functions of this package. For your convenience, it is better to be set to a sample rsf file in order to skip declaring it in input argument

\end{itemize}
%
%
\newpage
\section*{Structure}
Note that there are many parameters in this method that give more control over characterization process. However, to make it simple I have only included key parameters in function calls and the rest are set in the codes. So, if you need to have  more control over settings,l  you can access them inside \texttt{Main.py} file header.
\subsection*{Folders:}
The package consists of some folders as below:

\begin{description}
\item [Core] It holds main modules of package. Each module has special purposed functions inside. Here is a short list of exiting files in \textit{Core} folder and their functionality.

  \begin{verse}
    \texttt{\_\_init\_\_.py} : Initializes the import command in
    python

    \texttt{Misc.py}	: Miscellaneous functions for use in other files.

    \texttt{API.py} : Python interface to RSF for exchanging data.

    \texttt{Synthesize.py} : Module for generating synthetic data using fractional splines or Gaussian manifolds.

    \texttt{Cwt.py} : Module for continuous wavelet transform and other wavelet analysis.

    \texttt{Window.py} : Module for different window functions to be used in segmentation of events in seismic trace.

    \texttt{Manifold.py} : Module for generating Gaussian manifolds and handling them.

    \texttt{Steps.py} : Includes functions to do different steps of characterization as detection, segmentation, and estimation.

    \texttt{Main.py} : Includes characterization function which implements our new method for input seismic trace.

    \texttt{Show.py} : Carries functions for plotting results.
  \end{verse}


\item [Data] This folder is default folder for searching input files
  unless \texttt{CHARM\_Data} is set as explained in installation
  section.

\item [Results] This folder is default folder for storing outputs of
  characterization unless \texttt{CHARM\_Results} is set as explained
  in installation section.

\item [Doc] It carries documentation files for the package.

\item [pydata] This is where generated python data files by package
  are stored for future use unless \texttt{CHARM\_pydata} is set as
  explained in installation section.

\item [Demos] Any saved figure through using this package will sits in
  this folder unless \texttt{CHARM\_Demos} is set as explained in
  installation section.It also contains a simple demo to show how
        the main steps are used.

\end{description}
%
%
\subsection*{Files:}
\textit{Charm} folder also contains some other files as below:
\begin{description}
\item \texttt{\_\_init\_\_.py} initializes the import command in
  python and sets global variables for the package.
\item  \texttt{setup.py} helps user to install package as explained before.
\item \texttt{README}  quick guide user 
\item \texttt{LICENSE} license agreement information
\item \texttt{GNU} GNU general public license
\item  \texttt{SConstruct} \textit{scons} script which runs \texttt{sfchar.py}. Input and output arguments has to be set as explained in the files comments.
\item \texttt{sfchar.py} a wrapper for function
  \textit{char}(responsible for the analysis of seismic traces for
  characterization and extracting attributes) in \texttt{Main.py} file
  to be called with standard I/O in Unix terminal as explained later
  in \textit{Using package} section. This will implement the method on
  2-D seismic sections as input.
\item  \texttt{sfshow.py} responsible for plotting \textit{rsf} files from Unix terminal with python imaging library. It is able to plot either ordinary 2-D \textit{rsf} files or the 3-D \textit{rsf} files generated by \texttt{sfchar.py}. See the file header for more details about syntax and arguments.
\end{description}
%
%
\section*{Key variables in Python environment}

This section explains important variables that are passed in the
output dictionary of characterization function (\textit{char} in
\texttt{Main.py}) which is responsible for full analysis of seismic
traces. Remind that this function is also able to generate synthetic
data. For more information check \texttt{Main.py} file. The variables
dealing with trace and generated trace with estimation are as below:
\begin{description}
\item \texttt{N}	: length of trace
\item \texttt{trace} : seismic trace as a row vector
\item \texttt{trace\_det} : trace formed by superposition of windowed
  events
\item \texttt{trace\_est} : trace formed by superposition of
  estimations of each event
\item \texttt{err} : array consisting of estimation errors for each event.
\item \texttt{events} : a matrix whose rows are events in seismic
  trace.
\item \texttt{events\_det}: a matrix whose rows are segmented events
  of seismic trace.
\item \texttt{masks} : a matrix whose rows has corresponding mask used
  to segment each event.

\end{description}
%
\subsubsection*{Attribute matrices:}
Variables with the form of attrib\_xxx are matrices of attributes of
events found in the seismic trace. Each row of matrix has attributes
(location,scale, singularity order, and instantaneous phase) of one
events in trace.
\begin{description}
\item \texttt{attrib\_org} : original attributes for synthetic trace.
\item \texttt{attrib\_det} : attributes after detection step.
\item \texttt{attrib\_est} : attributes after estimation step.
\end{description}
%
Variables with the form of attrib\_vect are matrices, where number of
columns is equal to length of seismic trace. The rows show actual
values of amplitude, scale, singularity order, and phase components at
location of events and set to null elsewhere.
\begin{description}
\item \texttt{attrib\_vect} : Original attributes in case of synthetic data.
\item \texttt{attrib\_vect\_est} : Estimated attributes.
\end{description}
%
\subsubsection*{Python data files (pyd extension):}
These are generated python data files by package(\textit{char}
function) stored for future reload.
\begin{description}
\item \texttt{synt\_model.pyd} : Contains a dictionary of parameters
  for synthetic seismic data.
\item \texttt{real\_model.pyd} :  Contains real seismic trace.
\item \texttt{waves .pyd} : Contains a dictionary of source wavelet
  and CWT parameters.
\item \texttt{attrib\_det.pyd} : Contains matrix of attributes after
  detection step.
\item \texttt{attrib\_est.pyd} : Contains matrix of attributes after
  estimation step.
\item \texttt{events\_det .pyd} : A matrix of traces, each containing
  one windowed event.
\item \texttt{events\_est.pyd} : A matrix of traces, each containing
  one estimated event.
\item \texttt{trace\_est.pyd} : Contains superposition of estimated
  events which should be similar to actual trace.
\end{description}
%
For more information about these data files refer to \texttt{pydataFiles.txt}
%
%
\section*{How to use package}
\subsubsection*{Python:}
Go into your python interpreter. Type "import Charm" at your python
command prompt. As a demo, you may run following command in python
after importing the package.
\begin{verse}
  \texttt{Charm.char(type='new',user=1)}
\end{verse}
%
You may also try more detailed things while using \emph{detect,
  segment, and BFGSestimate} functions from the package. All these
functions are well-documented and easy to follow. To get help for any
function you can use \emph{help(name of function)}.

\subsubsection*{Unix terminal with standard input and output:}
Use following commands in terminal to analyze seismic data and show
results.
\begin{verse}
  \texttt{./sfchar.py <input.rsf >output.rsf args=...}

  \texttt{./sfshow.py <output.rsf           args=...}
\end{verse}
%
Arguments are optional and will be set to default if not provided. For more details about arguments check the scripts header for documentation.\\[20pt]
%
%
\href{mailto:mmaysami@eos.ubc.ca}{\textbf{Mohammad Maysami}}\\
\href{http://www.eos.ubc.ca/~mmaysami}{http://www.eos.ubc.ca/$\sim$mmaysami}\\
\href{http://slim.eos.ubc.ca}{Seismic Laboratory for Imaging and Modeling~(SLIM)}\\ 
\href{http://www.ubc.ca}{University of British Columbia (UBC)}
\end{document} 
